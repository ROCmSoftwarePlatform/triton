\ExplSyntaxOn
\NewExpandableDocumentCommand{\bitwiseXor}{mm}
 {
  \recuenco_bitwise_xor:nn { #1 } { #2 }
 }

\cs_new:Nn \recuenco_bitwise_xor:nn
 {
  \int_from_bin:e
   {
    \__recuenco_bitwise_xor:ee { \int_to_bin:n { #1 } } { \int_to_bin:n { #2 } }
   }
 }
\cs_generate_variant:Nn \int_from_bin:n { e }

\cs_new:Nn \__recuenco_bitwise_xor:nn
 {
  \__recuenco_bitwise_xor_binary:ee
   {
    \prg_replicate:nn
     {
      \int_max:nn { \tl_count:n { #1 } } { \tl_count:n { #2 } } - \tl_count:n { #1 }
     }
     { 0 }
     #1
   }
   {
    \prg_replicate:nn
     {
      \int_max:nn { \tl_count:n { #1 } } { \tl_count:n { #2 } } - \tl_count:n { #2 }
     }
     { 0 }
     #2
   }
 }
\cs_generate_variant:Nn \__recuenco_bitwise_xor:nn { ee }

\cs_new:Nn \__recuenco_bitwise_xor_binary:nn
 {
  \__recuenco_bitwise_xor_binary:w #1;#2;
 }
\cs_generate_variant:Nn \__recuenco_bitwise_xor_binary:nn { ee }

\cs_new:Npn \__recuenco_bitwise_xor_binary:w #1#2;#3#4;
 {
  \int_abs:n { #1-#3 }
  \tl_if_empty:nF { #2 } { \__recuenco_bitwise_xor_binary:w #2;#4; }
 }

\ExplSyntaxOff

\newcommand{\drawTensorLayoutGlobalMem}[4]{
  %%
  %% Draw tensor layout in global memory without any swizzling
  %%
  %% TL: pre defined top-left coordinates of the tensor in global memory
  %% \elemH: The height of each element
  %% \bsize: The width of each byte
  %% \Colors: a pre defined array of 16 colors
  %% \trans: 1 for K x N and 0 for M x K
  %%
  %% #1: rowName
  %% #2: colName
  %% #3: rowSize, i.e. number of rows
  %% #4: colSize, i.e. number of cols

  \pgfmathsetmacro{\rowSize}{#3}
  \pgfmathsetmacro{\colSize}{#4}
  %% decide how many rows to draw
  \ifthenelse{\trans=0}{
    % non-trans case
    \pgfmathsetmacro{\maxRowId}{\mfmaNonKDim-1}
  }{
    % trans case
    \pgfmathsetmacro{\maxRowId}{32/\bytesPerElem-1}
  }
  \pgfmathsetmacro{\elemsPerVec}{\vec}

  \pgfmathsetmacro{\vecInCol}{\colSize/\elemsPerVec}
  \pgfmathsetmacro{\maxColVecId}{\vecInCol-1}

  \foreach \gp in {0,...,\maxColVecId}{
    \pgfmathsetmacro{\gpCol}{int(mod(\gp, 16))}
    \pgfmathsetmacro{\vecColor}{\Colors[\gpCol]}
    \pgfmathsetmacro{\colStart}{int(\gp*\elemsPerVec)}
    \pgfmathsetmacro{\colEnd}{int(\colStart+\elemsPerVec-1)}
    \foreach \row in {0,...,\maxRowId}{
      \coordinate (vec TL) at ($(TL)+(\gp*\vecInBytes*\bsize, -\row*\elemH)$);
      \draw [ultra thin, fill=\vecColor] (vec TL) rectangle ++(\vecInBytes*\bsize, -\elemH)
      node [pos=.5, scale=.6*\bankLabelScale*\scale, white] {#1\row,#2\colStart:\colEnd};
    }
  }
  %% draw dims
  \def\gap{3}
  \pgfmathsetmacro{\drawRow}{\maxRowId*\elemH+\gap*\elemH+\elemH}
  \pgfmathsetmacro{\drawCol}{\vecInCol*\vecInBytes*\bsize}
  \draw [ultra thick] (TL) rectangle ++(\drawCol, -\drawRow);
  \node [scale=\scale, above, rotate=90] at ($(TL)+(0, -0.5*\drawRow)$) {block\_#1 = \rowSize};
  \node [scale=\scale, above] at ($(TL)+(0.5*\drawCol, 0)$) {block\_#2 = \colSize$\times$\dtype};
  \node [scale=\scale, rotate=90] at ($(TL)+(0.5*\colSize*\bytesPerElem*\bsize, -\drawRow+.5*\gap*\elemH)$) {$\ldots$};
}


\newcommand{\drawLDSDiagram}[3]{
  %%
  %% Draw the diagram of LDS without any data
  %%
  %% Pre-defined variables
  %% TL: top-left coordinates of first elements in LDSaccess
  %% bsize: size of a byte
  %% mfmaNonKDim
  %% K
  %% bytesPerElem
  %%
  %% #1: number of banks
  %% #2: rows of tensor plotted
  %% #3: columns of tensor plotted

  \pgfmathsetmacro{\banks}{#1}
  \pgfmathsetmacro{\rows}{#2}
  \pgfmathsetmacro{\cols}{#3}
  \pgfmathsetmacro{\maxBankId}{\banks-1}
  \pgfmathsetmacro{\tensorHeight}{\rows*\cols*\bytesPerElem/4/\banks*\elemH}
  \def\gapT{4}
  \def\gapB{2}
  \pgfmathsetmacro{\LDSHeight}{\tensorHeight+\gapT*\elemH+\gapB*\elemH}
  \coordinate (LDS TL) at ($(TL)+(0, \gapT*\elemH)$);
  \foreach \bank in {0,...,\maxBankId}{
    \coordinate (bank TL) at ($(LDS TL)+(\bank*4*\bsize, 0)$);
    \draw [ultra thick] (bank TL) rectangle ++(4*\bsize, -\LDSHeight);
    \node [scale=.6*\bankLabelScale*\scale, below, align=center] at ($(bank TL)+(2*\bsize,0)$) {bank\\\bank};
    \node [scale=0.8*\bankLabelScale*\scale, rotate=90] at ($(TL)+(2*\bsize+\bank*4*\bsize, -\tensorHeight-0.5*\gapB*\elemH)$) {$\ldots$};
  }
  \node [scale=\scale, above] at ($(TL)+(0.5*\banks*4*\bsize, 4*\elemH)$) {LDS \banks\ banks};
}

\newcommand{\drawHighlightedAccess}[3]{
  %% Highlight the vectors if \tid < \threshold
  %%
  %% Predefined variables
  %% vec TL: top-left of the current vector
  %% elemH, bsize, vecInBytes, bankLabelScale
  %%
  %% #1: tid
  %% #2: threshold
  %% #3: label in vector

  \pgfmathsetmacro{\tid}{#1}
  \pgfmathsetmacro{\threshold}{#2}
  \def\bWidth{0.02}

  \ifthenelse{\tid < \threshold}{
    \ifthenelse{\vecInBits=128}{\def\ratio{0.5}}{\def\ratio{1}}
    \draw [thick, draw=white, fill=\vecColor] ($(vec TL)+(\bWidth, -\bWidth)$) rectangle ++(\vecInBytes*\bsize-2*\bWidth, -\elemH+2*\bWidth);
    \ifthenelse{\vecInBits=128}{
      \node [scale=.6*\bankLabelScale*\scale, white] at ($(vec TL)+(\ratio*\vecInBytes*\bsize, -0.5*\elemH)$) {#3};
    }{
      \node [scale=.6*\bankLabelScale*\scale, white, left] at ($(vec TL)+(\ratio*\vecInBytes*\bsize, -0.5*\elemH)$) {#3};
    }
    \node [scale=.5*\bankLabelScale*\scale, right] at ($(vec TL)+(0, -0.5*\elemH)$) {\textbf{t\tid}};
  }{}
}

\newcommand{\drawCoalescedGRAccess}[3]{
  %% Highlight the vectors in original tile if \tid < \threshold
  %%
  %% Predefined variables
  %% tile TL: top-left of the original tile
  %% gp: vector group id along K dim
  %% row: row index
  %% elemH, bsize, vecInBytes, bankLabelScale
  %%
  %% #1: tid
  %% #2: threshold
  %% #3: label in vector

  \pgfmathsetmacro{\tid}{#1}
  \pgfmathsetmacro{\threshold}{#2}
  \def\bWidth{0.02}

  \ifthenelse{\tid < \threshold}{
    \ifthenelse{\vecInBits=128}{\def\ratio{0.5}}{\def\ratio{1}}
    \coordinate (tile vec TL) at ($(tile TL)+(\gp*\vecInBytes*\bsize, -\row*\elemH)$);
    \draw [thick, draw=white, fill=\vecColor] ($(tile vec TL)+(\bWidth, -\bWidth)$) rectangle ++(\vecInBytes*\bsize-2*\bWidth, -\elemH+2*\bWidth);
    \ifthenelse{\vecInBits=128}{
      \node [scale=.6*\bankLabelScale*\scale, white] at ($(tile vec TL)+(\ratio*\vecInBytes*\bsize, -0.5*\elemH)$) {#3};
    }{
      \node [scale=.6*\bankLabelScale*\scale, white, left] at ($(tile vec TL)+(\ratio*\vecInBytes*\bsize, -0.5*\elemH)$) {#3};
    }
    \node [scale=.5*\bankLabelScale*\scale, right] at ($(tile vec TL)+(0, -0.5*\elemH)$) {\textbf{t\tid}};
  }{}
}

\newcommand{\drawLDSLayoutAndAccess}[6]{
  %%
  %% Draw tensor layout in LDS with swizzling
  %%
  %% TL: pre defined top-left coordinates of the tensor in global memory
  %% \elem: per defined variable
  %% \Colors: a pre defined array of 16 colors
  %%
  %% The following three arguments are expected to be pre defined
  %% vec: number of elements in a group
  %% trans
  %% useMfmaTransLD
  %%
  %% #1: hasSwizzle, 0 means no swizzling and no padding,
  %%                 1 means optimal swizzling
  %%                 2 means padding
  %% #2: access mode, 0 means draw nothing, 1 means ds_read, 2 means ds_write
  %% #3: number of banks
  %% #4: rowLabel
  %% #5: colLabel
  %% #6: colSize

  \pgfmathsetmacro{\hasSwizzle}{#1}
  \pgfmathsetmacro{\accessMode}{#2}
  \pgfmathsetmacro{\numVecCol}{\colSize/\vec}
  \pgfmathsetmacro{\banks}{#3}
  \pgfmathsetmacro{\colSize}{#6}

  \ifthenelse{\trans=0}{
    \drawLDSDiagram{#3}{\mfmaNonKDim}{\colSize}
  }{
    \drawLDSDiagram{#3}{32/\bytesPerElem}{\colSize}
  }

  % number of elements per LDS row
  \pgfmathsetmacro{\elemsPerLDSRow}{int(\banks*4/\bytesPerElem)}
  % number of vectors per LDS row
  \pgfmathsetmacro{\vecsPerLDSRow}{\elemsPerLDSRow/\vec}
  % max vecId per tile row
  \pgfmathsetmacro{\maxColVecId}{\colSize/\vec-1}

  %% Parameters for ds_read
  % \vecInBytes: access width in bytes
  % \vecInBits: access width in bits (ds_read_b64 or ds_read_b128)
  % \numThreadsSameCycle: number of threads that will access LDS at the same cycle (8, 16, or 32)
  \pgfmathsetmacro{\vecInBits}{int(\vecInBytes*8)}
  \pgfmathsetmacro{\elemInBits}{int(\bytesPerElem*8)}
  \pgfmathsetmacro{\numThreadsSameCycle}{int(\banks*4/\vecInBytes)}
  \pgfmathsetmacro{\maxTid}{int(\numThreadsSameCycle-1)}

  %% Parameters for swizzling
  %% perPhase = ceil(elemsPerLDSRow / K)
  %% The number of the rows of the tensor that can share the same swizzling pattern
  \pgfmathsetmacro{\perPhase}{ceil(\elemsPerLDSRow/\colSize)}
  %% maxPhase: the total number of different swizzling patterns
  \ifthenelse{\hasSwizzle=0}{
    %% When swizzling is disabled
    \pgfmathsetmacro{\maxPhase}{1}
  }{
    %% When vec is small enough, we want 16/perPhase different swizzling patterns
    %% When vec is large, we can only have 64 / \vec different swizzling pattern at most
    \pgfmathsetmacro{\maxPhase}{min(min(\mfmaNonKDim,\numThreadsSameCycle)/\perPhase,\banks*4/\bytesPerElem/\vec)}
  }

  %% Draw the vectors according to the swizzling pattern
  \foreach \gp in {0,...,\maxColVecId}{
    \pgfmathsetmacro{\gpCol}{int(mod(\gp, 16))}
    \pgfmathsetmacro{\vecColor}{\Colors[\gpCol]}
    \pgfmathsetmacro{\colStart}{int(\gp*\elemsPerVec)}
    \pgfmathsetmacro{\colEnd}{int(\colStart+\elemsPerVec-1)}
    \foreach \row in {0,...,\maxRowId}{
      %% Compute some info of the current vec
      % global offset in unit of vec
      \pgfmathsetmacro{\offVec}{\row*\colSize/\vec+\gp}
      % which row of LDS
      \pgfmathsetmacro{\LDSRow}{int(\offVec/\vecsPerLDSRow)}
      % offset in the current LDS row in the unit of vec
      \pgfmathsetmacro{\LDSVecRaw}{int(mod(\offVec,\vecsPerLDSRow))}
      \pgfmathsetmacro{\phaseRaw}{int(\row/\perPhase)}
      \pgfmathsetmacro{\phase}{int(mod(\phaseRaw, \maxPhase))}
      \pgfmathsetmacro{\LDSVec}{\bitwiseXor{\LDSVecRaw}{\phase}}

      \coordinate (vec TL) at ($(TL)+(\LDSVec*\vecInBytes*\bsize, -\LDSRow*\elemH)$);
      \draw [ultra thin, fill=\vecColor] (vec TL) rectangle ++(\vecInBytes*\bsize, -\elemH)
      node [pos=.5, scale=.6*\bankLabelScale*\scale, white] {#4\row,#5\colStart:\colEnd};

      %% draw phase of each LDS row
      \pgfmathsetmacro{\lastVecId}{\vecsPerLDSRow-1}
      \ifthenelse{\LDSVec=\lastVecId}{
        \draw [ultra thin] ($(vec TL)+(\vec*\bytesPerElem*\bsize, -.5*\bsize)$) -- ++(\elemH, 0)
        node [scale=0.6*\bankLabelScale*\scale, right] {\phase};
      }{}

      %% For ds_read/write access patterns, we first decide the thread id that owns
      %% the current vector. And then we decide if the current vector is accessed
      %% at the first cycle according to thread id and access width

      % Draw ds_read
      \ifthenelse{\accessMode=1}{
        \ifthenelse{\trans=0}{
          %% K-contig case
          %%% compute thread id for current vec
          \pgfmathsetmacro{\tid}{int(\gp*\mfmaNonKDim+\row)}
          %%% draw ds_read instruction name
          \ifthenelse{\tid=0}{
            \node [scale=\scale, above right] at ($(TL)+(0, \gapT*\elemH)$)
            {ds\_read\_b\vecInBits\ (t0$\sim$t\maxTid)};}{}
          %%% highlight vector of the threads that will access LDS at the same cycle
          \drawHighlightedAccess{\tid}{\numThreadsSameCycle}{#4\row,#5\colStart:\colEnd}
        }{
          %% MN-contig case
          %%% This is further diverging according to whether mfma_transpose_load is used
          \ifthenelse{\useMfmaTransLD=0}{
            %%% Not use mfma_transpose_load instructions
            %%%% This is the current triton implementation of MN-contig case. We have
            %%%% - Threads can only load one element, i.e. ds_read_b16/b8 are used
            %%%% - 32 threads are accessing LDS at the same cycle
            %%%%   - if mfmaNonKDim == 32, they will only access row 0
            %%%%   - if mfmaNonKDim == 16, they will access row 0 and mfmaKWidth
            %%%% - no swizzling used for lds layout
            %%%% - vec is always 16 bytes

            \pgfmathsetmacro{\numGp}{\mfmaNonKDim/\vec}
            \pgfmathsetmacro{\maxElId}{\vec-1}
            \pgfmathsetmacro{\hasSecondRow}{int((32-\mfmaNonKDim)/16)}
            \pgfmathsetmacro{\secondRow}{\hasSecondRow*\mfmaKWidth}
            \pgfmathsetmacro{\tStart}{\colStart+int(\row/\mfmaKWidth)*16}
            \pgfmathsetmacro{\tEnd}{\tStart+\vec}
            \ifthenelse{\gp < \numGp}{
              \ifthenelse{\row = 0 \OR \row = \secondRow}{
                \foreach \el in {0,...,\maxElId}{
                  \pgfmathsetmacro{\tid}{int(\tStart+\el)}
                  %%%% Draw access in LDS
                  \draw ($(vec TL)+(\el*\bytesPerElem*\bsize, 0)$) rectangle ++(\bytesPerElem*\bsize, -\elemH)
                  node[scale=0.3*\bankLabelScale, pos=.5] {t\tid};
                  %%%% Draw access in original tile
                  \draw [opacity=0.5] ($(tile TL)+(\gp*\vecInBytes*\bsize+\el*\bytesPerElem*\bsize, -\row*\elemH)$) rectangle ++(\bytesPerElem*\bsize, -\elemH*\mfmaKWidth)
                  node[scale=0.3*\bankLabelScale, pos=0, below right] {t\tid};
                }
                \draw [ultra thin, fill=\vecColor, opacity=0.6] (vec TL) rectangle ++(\vecInBytes*\bsize, -\elemH)
                node [pos=.5, scale=.6*\bankLabelScale*\scale, white] {#4\row,#5\colStart:\colEnd};
              }{}
            }{}
            \ifthenelse{\gp = 0 \AND \row = 0}{
              \node [scale=\scale, above right] at ($(TL)+(0, \gapT*\elemH)$)
              {ds\_read\_b\elemInBits\ (t0$\sim$t31)};}{}
          }{
            %%% Use mfma_transpose_load instructions
            
          }
        }
      }{} %% End draw ds_read

      %% Draw ds_write
      \ifthenelse{\accessMode=2}{
        % compute thread id for current vec
        % Here we assume the following global load pattern:
        % - global/buffer_load_dwordx4, i.e. sizePerThread[1] = 128-bit
        % - CTA coverage will always cover all elements along the K dim first, i.e.
        %   sizePerThread[1]*threadsPerWarp[1] == K or
        %   (sizePerThread[1]*threadsPerWarp[1] < K and threadsPerWarp[0] == 1)
        \pgfmathsetmacro{\offBytes}{int(\row*\colSize*\bytesPerElem+\gp*\vecInBytes)}
        \pgfmathsetmacro{\tidRaw}{int(\offBytes/16)}
        \pgfmathsetmacro{\remTid}{int(mod(\offBytes,16))}
        \drawCoalescedGRAccess{\tidRaw}{\numThreadsSameCycle}{#4\row,#5\colStart:\colEnd}
        \ifthenelse{\remTid>0}{\pgfmathsetmacro{\tid}{int(\tidRaw+32)}}{\pgfmathsetmacro{\tid}{\tidRaw}}
        % draw ds_write instruction name
        \ifthenelse{\tid=0}{
          \node [scale=\scale, above right] at ($(TL)+(0, \gapT*\elemH)$)
          {ds\_write\_b\vecInBits\ (t0$\sim$t\maxTid)};}{}
        % highlight vector of the threads that will access LDS at the same cycle
        \drawHighlightedAccess{\tid}{\numThreadsSameCycle}{#4\row,#5\colStart:\colEnd}
      }{} %% End draw ds_write
      
    }
  }
  \node [scale=0.6*\bankLabelScale*\scale, above right] at($(TL)+(\banks*4*\bsize, 0)$) {phase};
}
